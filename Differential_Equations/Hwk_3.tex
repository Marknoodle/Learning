\documentclass[12pt, letterpaper]{article}
\usepackage[utf8]{inputenc}
\usepackage[letterpaper, margin=.5in]{geometry}

\title{Hwk 3}
\author{Mark Holcomb}
\date{March 2020}
\usepackage{roboto}
\usepackage{mathtools}


\begin{document}
\LARGE\textbf{Homework 3:}\\

\large\textbf{Problem 1}

\textbf{a)}

given: $P =\frac{R}{2L}$,\indent
       $\omega_o = \sqrt{\frac{1}{Lc}}$,\indent
       $L I'' + RI' + C^{-1}I = E'(t)$ \\

       Now we are going to take the value of $E(t)$ and find it's derivative
       
       \begin{center}
        $E(t) = E_o \sin(\omega t)$\\
        $E'(t) = \omega E_o \cos(\omega t)$
       \end{center}
       
       This gives us that our original equation is actually
       \begin{center}
        $L I'' + RI' + C^{-1}I = \omega E_o \cos(\omega t)$
       \end{center}

       Now we can take the particular solution, $X_p$, from the book and plug our values in
       \begin{center}
        $X_p = [\frac{\omega E_o}{L \sqrt{(2 \omega p)^2 + (\omega_o^2 - \omega^2)^2}}] \sin(\omega t - y)$
       \end{center}

\textbf{b)}

To find the amplitude, let's first consider the form $y = A sin(B(x + C)) + D$\\

Where A is the amplitude, we can see from our equation from a) that in our case,
\begin{center}
    $A = \frac{\omega E_o}{L \sqrt{(2 \omega p)^2 + (\omega_o^2 - \omega^2)^2}}$
\end{center}

\textbf{c)}

An amplitude is maximized when it's derivative(slope) is 0, or when $A' = 0$

\begin{center}
    $A' = $
\end{center}

\end{document}

